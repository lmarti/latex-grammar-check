\documentclass{beamer}

\usecolortheme[named=blue]{structure}

\mode<presentation>
{
  \usetheme{Warsaw}
  \setbeamercovered{transparent}
  \setbeamertemplate{items}[ball]
  \setbeamertemplate{theorems}[numbered]
  \setbeamertemplate{footline}[frame number]
}

\usepackage{beamerthemesplit}
\usepackage{graphics}
\usepackage{graphicx}
\usepackage{hyperref}

\title
  [A Simple Presentation using the Beamer Class]
  {Simple~Presentation~Using~the~Beamer~Class}
\author[Webster,Gunzburger]{%
  Clayton~Webster\inst{1} \and
  Max Gunzburger\inst{2}}
\institute[Florida State University]{
  \inst{1}%
  Department of Mathematics and School for Computational Science\\
  Florida State University
  \and
  \inst{2}%
  School for Computational Science\\
  Florida State University}
\date[DLT 2004]{SIAM Conference, 2004}
\subject{Computational Sciences}

\pgfdeclaremask{fsu}{fsu_logo}
\pgfdeclareimage[mask=fsu,width=1cm]{fsu-logo}{fsu_logo}

\logo{\vbox{\vskip0.1cm\hbox{\pgfuseimage{fsu-logo}}}}

\begin{document}
%-----------------------------------------------------------------------------80
  \frame
  {
    \titlepage
  }
%-----------------------------------------------------------------------------80
  \section*{Outline}
%-----------------------------------------------------------------------------80
  \frame
  {
    \frametitle{Outline of Topics}

    \tableofcontents
  }
%-----------------------------------------------------------------------------80
  \section{Lists}
%-----------------------------------------------------------------------------80
  \frame
  {
    \frametitle{"Dynamic" Lists}

    With a few extra marks, you can have Beamer march
    through each item on a list one at a time.
  }
%-----------------------------------------------------------------------------80
  \frame
  {
    \frametitle{A Sample List}

    \begin{itemize}
    \item<1-> Normal LaTeX class.
    \item<2-> Easy overlays.
    \item<3-> No external programs needed.      
    \end{itemize}
  }
%-----------------------------------------------------------------------------80
\frame
{
  \frametitle{Displayed Text}

  The {\bf{block}} command can be used to highlight a statement.
  \begin{block}{A famous quote}
    {\it{``Prime numbers are more than any assigned multitude of prime numbers.''}}\\
    Euclid
  \end{block}
  Because of Aristotle's objections, Euclid did {\bf{not}} say:
  \begin{block}{}
    The list of all prime numbers is infinite.
  \end{block}

}
%-----------------------------------------------------------------------------80
\frame
{
  \frametitle{Displayed Text}

  The {\bf{theorem}} command can be used for theorems.  If the appropriate
  beamer environment is set up, the theorems will be numbered.
  \begin{theorem}
    {\it{``Prime numbers are more than any assigned multitude of prime numbers.''}}\\
    Euclid
  \end{theorem}
  Because of Aristotle's objections, Euclid did {\bf{not}} say:
  \begin{theorem}
    The list of all prime numbers is infinite.
  \end{theorem}

}
%-----------------------------------------------------------------------------80
  \section{Graphics}
%-----------------------------------------------------------------------------80
  \frame
  {
    \frametitle{Including Graphics}

    Most graphics files can be converted to {\bf .png} format.
    Such files are then easily included in your presentation.
    
    \begin{tiny}
    You can change text size with the {\bf{tiny}} environment, for example.
    \end{tiny}
  }
%-----------------------------------------------------------------------------80
  \frame
  {
    \frametitle{PNG Graphics}

    A graphics file is included by using the {\bf figure} environment,
    and inside of that the {\bf includegraphics} command.
    \begin{figure}
      \scalebox{0.30}
      {
        \includegraphics{pengbrew.png}
      }
    \end{figure}
    {\it{In Beamer, when you place a figure in the text, that's where it shows
    up in the slide.}}  Let's repeat the figure at a smaller size.
    \begin{figure}
      \scalebox{0.20}
      {
        \includegraphics{pengbrew.png}
      }
    \end{figure}
  }
%-----------------------------------------------------------------------------80
  \frame
  {
    \frametitle{PNG Graphics}

    You can also place figures side by side.
    \begin{figure}
      \includegraphics[width=0.30\textwidth]{pengbrew.png}
      \includegraphics[width=0.10\textwidth]{pengbrew.png}
      \includegraphics[width=0.50\textwidth]{pengbrew.png}
    \end{figure}
  }
%-----------------------------------------------------------------------------80
 \frame
  {
    \frametitle{PNG Graphics}

    Beamer allows you to include mathematical formulas in your presentation
    as part of your text.  For example, you can use the {\bf{displaymath}}
    environment:
    \begin{displaymath}
      || u^h - u || <= ||u^h - \mathcal{P}^h(u)|| + ||\mathcal{P}^h(u) - u||
    \end{displaymath}
  }
%-----------------------------------------------------------------------------80
  \frame
  {
    \frametitle{PNG Graphics}

    The previous graphic had a background that almost matched the
    slide.  A more typical case is shown here.
    The graphics figure simply shows up as a rectangle imposed on
    the slide.
    \begin{figure}
      \scalebox{0.35}
      {
        \includegraphics{bell_206.png}
      }
    \end{figure}
  }
%-----------------------------------------------------------------------------80
  \section{Verbatim Text}
%-----------------------------------------------------------------------------80
  \begin{frame}[fragile]\
  {
    \frametitle{Using Verbatim}

    Sometimes you just want something to show up exactly as you typed it.
    You can use Latex's {\bf{verbatim}} environment for this.  But a Beamer
    frame containing the {\bf{verbatim}} command must use a more complicated
    scheme to begin and end the frame.  The Beamer frame must begin with:
\begin{verbatim}
  \begin{frame}[fragile]\
\end{verbatim}
and end with an explicit:
\begin{verbatim}
  \end{frame}
\end{verbatim}
  Look at the source for this slide to see an example.
  }
\end{frame}
%-----------------------------------------------------------------------------80
  \begin{frame}[fragile]\
  {
    \frametitle{Using Semiverbatim}

    Beamer includes a {\bf{semiverbatim}} environment which allows you to
    place text exactly where you want it, except that you can include a few
    simple style commands, such as for color.  Such frames must begin and end
    the same way {\bf{verbatim}} frames do.
\begin{semiverbatim}
  Day       Cost  Comment
  Tuesday   \$7   {\it{Too much!}}
  Wednesday \$14  {\tiny{I got sick}}
  Thursday  {\alert{\$45}}  Luckily, I didn't have to pay.
\end{semiverbatim}
  }
\end{frame}
%-----------------------------------------------------------------------------80
  \begin{frame}[fragile]\
  {
    \frametitle{Using Semiverbatim}

    Semiverbatim is one way to display a sample program, using the
    {\bf{alert}} command to highlight an important line.   If your program
    uses curly brackets, you'll have to ``escape'' them within the
    semiverbatim environment.
\begin{tiny}
\begin{semiverbatim}
  void swap ( int *a, int *b )
  \{
    int c; 
    c = *a;  
    *a = *b;
    *b = {\alert{*c}};   <-- {\it{This is where you made the mistake!}}
    return;
  \} 
\end{semiverbatim}
\end{tiny}
  }
\end{frame}
%-----------------------------------------------------------------------------80
  \section{Hyperlinks}
%-----------------------------------------------------------------------------80
  \frame
  {
    \frametitle{Active Hyperlinks}

    Inside of your presentation,  you can use the {\bf{href}} environment
    to define a hyperlink.  Clicking on the displayed text will then invoke
    a browser and go to the specified link.
\vskip 0.1in
    For example, for more information on the Beamer Class, you can click on
    \href{https://bitbucket.org/rivanvx/beamer/wiki/Home}{{\alert{the Beamer web site}}}.

  }
%-----------------------------------------------------------------------------80
\end{document}
